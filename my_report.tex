\documentclass[man,floatsintext]{apa6}
\usepackage{lmodern}
\usepackage{amssymb,amsmath}
\usepackage{ifxetex,ifluatex}
\usepackage{fixltx2e} % provides \textsubscript
\ifnum 0\ifxetex 1\fi\ifluatex 1\fi=0 % if pdftex
  \usepackage[T1]{fontenc}
  \usepackage[utf8]{inputenc}
\else % if luatex or xelatex
  \ifxetex
    \usepackage{mathspec}
  \else
    \usepackage{fontspec}
  \fi
  \defaultfontfeatures{Ligatures=TeX,Scale=MatchLowercase}
\fi
% use upquote if available, for straight quotes in verbatim environments
\IfFileExists{upquote.sty}{\usepackage{upquote}}{}
% use microtype if available
\IfFileExists{microtype.sty}{%
\usepackage{microtype}
\UseMicrotypeSet[protrusion]{basicmath} % disable protrusion for tt fonts
}{}
\usepackage{hyperref}
\hypersetup{unicode=true,
            pdftitle={Affect, Cognition and Personality, Re-Examining the Competing Components of Subjective},
            pdfauthor={Aaron L. Willcox},
            pdfkeywords={subjective wellbeing, personality, happiness},
            pdfborder={0 0 0},
            breaklinks=true}
\urlstyle{same}  % don't use monospace font for urls
\usepackage{longtable,booktabs}
\usepackage{graphicx,grffile}
\makeatletter
\def\maxwidth{\ifdim\Gin@nat@width>\linewidth\linewidth\else\Gin@nat@width\fi}
\def\maxheight{\ifdim\Gin@nat@height>\textheight\textheight\else\Gin@nat@height\fi}
\makeatother
% Scale images if necessary, so that they will not overflow the page
% margins by default, and it is still possible to overwrite the defaults
% using explicit options in \includegraphics[width, height, ...]{}
\setkeys{Gin}{width=\maxwidth,height=\maxheight,keepaspectratio}
\IfFileExists{parskip.sty}{%
\usepackage{parskip}
}{% else
\setlength{\parindent}{0pt}
\setlength{\parskip}{6pt plus 2pt minus 1pt}
}
\setlength{\emergencystretch}{3em}  % prevent overfull lines
\providecommand{\tightlist}{%
  \setlength{\itemsep}{0pt}\setlength{\parskip}{0pt}}
\setcounter{secnumdepth}{0}
% Redefines (sub)paragraphs to behave more like sections
\ifx\paragraph\undefined\else
\let\oldparagraph\paragraph
\renewcommand{\paragraph}[1]{\oldparagraph{#1}\mbox{}}
\fi
\ifx\subparagraph\undefined\else
\let\oldsubparagraph\subparagraph
\renewcommand{\subparagraph}[1]{\oldsubparagraph{#1}\mbox{}}
\fi

%%% Use protect on footnotes to avoid problems with footnotes in titles
\let\rmarkdownfootnote\footnote%
\def\footnote{\protect\rmarkdownfootnote}


  \title{Affect, Cognition and Personality, Re-Examining the Competing Components
of Subjective}
    \author{Aaron L. Willcox\textsuperscript{1}}
    \date{}
  
\shorttitle{Competing Components of Subjective Wellbeing}
\affiliation{
\vspace{0.5cm}
\textsuperscript{1} Deakin University}
\keywords{subjective wellbeing, personality, happiness\newline\indent Word count: 2955}
\usepackage{csquotes}
\usepackage{upgreek}
\captionsetup{font=singlespacing,justification=justified}

\usepackage{longtable}
\usepackage{lscape}
\usepackage{multirow}
\usepackage{tabularx}
\usepackage[flushleft]{threeparttable}
\usepackage{threeparttablex}

\newenvironment{lltable}{\begin{landscape}\begin{center}\begin{ThreePartTable}}{\end{ThreePartTable}\end{center}\end{landscape}}

\makeatletter
\newcommand\LastLTentrywidth{1em}
\newlength\longtablewidth
\setlength{\longtablewidth}{1in}
\newcommand{\getlongtablewidth}{\begingroup \ifcsname LT@\roman{LT@tables}\endcsname \global\longtablewidth=0pt \renewcommand{\LT@entry}[2]{\global\advance\longtablewidth by ##2\relax\gdef\LastLTentrywidth{##2}}\@nameuse{LT@\roman{LT@tables}} \fi \endgroup}


\usepackage[titles]{tocloft}
\cftpagenumbersoff{figure}
\renewcommand{\cftfigpresnum}{\itshape\figurename\enspace}
\renewcommand{\cftfigaftersnum}{.\space}
\setlength{\cftfigindent}{0pt}
\setlength{\cftafterloftitleskip}{0pt}
\settowidth{\cftfignumwidth}{Figure 10.\qquad}

\authornote{

Correspondence concerning this article should be addressed to Aaron L.
Willcox, . E-mail:
\href{mailto:awillcox@deakin.edu.au}{\nolinkurl{awillcox@deakin.edu.au}}}

\abstract{
Subjective wellbeing (SWB) is described as the experience and
perceptions of an individual captured through feelings and thoughts.
This studied re-rested the components of affect, cognition and
personality on SWB. The study sampled 101 Australian residents aged
between 18 and 72 years old (\emph{M} = 35.25, SD = 10.65), recruited
through undergraduate students, shared via an online survey. Subjective
wellbeing scores were outside the normative range of 73.34 to 76.43
points (M = 68.80, SD = 10.50). Hierarchical regression found that at
step one, affect was the main component of SWB explaining 57\% of the
variance in SWB. Step two and three showed Cognition and personality
only weakly predicted SWB, respectively. Change effects of cognition
were not significant and further regression analysis revealed that
Neuroticism takes control when the model is in homeostatic defeat with
affect and neuroticism accounting for 66\% of the variance in SWB.
Current findings suggest that affect is the main driver of SWB, not
cognition when the individual is experiencing challenging circumstances.
Future studies could explore the limitations discussed herein.


}

\usepackage{amsthm}
\newtheorem{theorem}{Theorem}[section]
\newtheorem{lemma}{Lemma}[section]
\theoremstyle{definition}
\newtheorem{definition}{Definition}[section]
\newtheorem{corollary}{Corollary}[section]
\newtheorem{proposition}{Proposition}[section]
\theoremstyle{definition}
\newtheorem{example}{Example}[section]
\theoremstyle{definition}
\newtheorem{exercise}{Exercise}[section]
\theoremstyle{remark}
\newtheorem*{remark}{Remark}
\newtheorem*{solution}{Solution}
\begin{document}
\maketitle

\section{Introduction}\label{introduction}

Subjective wellbeing (SWB) is the term applied to explain an
individual's experience and perceptions of how they evaluate their life
across various domains. It is subjective in that it captures the
individuals cognitive and affective components in how they think and
feel in areas such as health, work and relationships. How these
components relate and merge to form an overall measure of wellbeing can
help determine what aspects of an individual's lives are driving how
they feel. Diener, E. Lucas, \& Oishi, 2018 argue that SWB is
predominately driven by a cognitive component and there is no single
overarching model of wellbeing. However, R. Cummins (2010) proposed a
neurophysiological model of SWB analogous to how body temperature is
regulated.

Regulation of wellbeing should fall within a normative range of 73.43 -
76.43 points and values outside of this range may indicate problems.
This suggests that the subjective-ness nature of wellbeing is innate and
generally positive, in contrast to Diener et al., 2018, external factors
only play a small role in happiness (Lucas, 2008). Furthermore, the
individual can adapt to challenging circumstances and return to a state
of normal functioning (Richardson, Fuller Tyszkiewicz, Tomyn, \&
Cummins, 2016). However, the extent to what components drive the model
and which is the strongest, is still under debate.

Current theories suggest that affective factors, such as a person's mood
and emotions, is the dominant component in the model (Davern, Cummins,
\& Stokes, 2007). SWB is generally positive and maintained by an
inherent set-point that is managed through an evolutionary system called
Homeostatically Protected Mood (HPMood; R. Cummins, Li, Wooden, and
Stokes (2014)). When this system is disrupted through challenging
experiences, the set-point is expected to drop below the average and
experience negative affects rather than positive ones (Capic, Li, \&
Cummins, 2018). This suggests that an individual is primarily controlled
by how they feel however, there is considerable variation in how an
individual adapts to challenges and therefore, SWB levels may change
(Lucas, 2008).

One explanation of these variations is that of personality. Costa \&
McCrae's (1980) theory of personality demonstrated how individuals
differ on SWB set-point scores on stable traits of extraversion and
neuroticism. Extraverts rated higher on SWB scores than that of
introverts and, neuroticism scores were lower than that of emotionally
stable individuals. For example, extraverted individuals tend to create
better social experiences which can lead to improved wellbeing (Harris
et al., 2017). In contrast, neuroticism can act as a mediator between
extraversion and wellbeing leading to lower SWB (Fadda \& Scalas, 2016).
Personality traits may explain why individuals experience life in
difference ways and how this influences their wellbeing (Pollock, Noser,
Holden, \& Zeigler-Hill, 2016).

However, these examples rely on stable personality traits, challenging
life circumstances may destabilise traits and cognitive appraisals may
take a salient role. Including a cognitive component may help explain
how personality traits differ according to evaluations of satisfaction,
and account for underlying mental structures used. One theory to capture
these evaluations is that of Multiple discrepancies theory (MDT; A. C.
Michalos (1985)). The hypothesis is that an individual's overall net
satisfaction is a function of a perception of ideals between past,
present and future desires (A. Michalos, 2017). Jovanovic (2011)
demonstrated that personality doesn't influence the cognitive component
of wellbeing however, neuroticism does influence the affective
component. Suggesting that, when personality traits are destabilized the
affective component takes control.

Each of these components contribute distinct mechanisms in explaining
SWB. Davern et al. (2007) tested a model by regressing 5 traits of
personality, with affect and cognition on a broad group of adult
Australians. Their study demonstrated that the joint contribution of the
cognitive-affective model explained 90\% of the variance in SWB.
Furthermore, personality traits were a weaker contributor than that of
cognition and affect; Extraversion and Agreeableness traits were the
only significant correlates of SWB. Blore, Stokes, Mellor, Firth, and
Cummins (2011) replicated this design with only two personality traits
of Neuroticism and Extraversion. The study demonstrated a purely
affective model explaining 66\% of the variance in SWB and furthermore,
cognition and the cognitive-affective model proved to be a poor fit to
SWB model. Tomyn and Cummins (2011) replicated Davern et al.'s (2007)
design in a group of Australian High school students and found that
affect was the better fit then either personality or cognition driven
model explaining 80\% of the variance in SWB.

While these findings highlight affect as the dominant component, there
were some reliability issues worth examining. (Blore et al., 2011) and
Tomyn and Cummins (2011) utilized the Ten Item Personality index (TIPI).
The TIPI is a short form personality scale designed to perform poorly on
reliability factor analysis (Gomez, Allemand, \& Grob, 2012).
Furthermore, the affect scale used in the Blore et al.'s (2011) was
constructed purely for their study and no validity was reported. While
Davern et al.'s (2007) study was comprehensive, no internal validity
measurements were reported and the NEO Five Factory Inventory used in
the study has known, internal structural issues (Rosellini \& Brown,
2011).

The present study aimed to firstly, retest the relative components of
affect, cognition and personality for predicting individual differences
in subjective wellbeing. Secondly, to address the limitations from Blore
et al., (2011), Davern et al., (2007) and Tomyn \& Cummins (2011), by
utilizing a more reliable personality inventory with all five
personality traits. This study hypothesized that a) mean scores of SWB
will fall within the normative range; b) affect, cognition and
personality will predict SWB and; c) affect will have the strongest,
unique contribution on predicting SWB.

\section{Method}\label{method}

\subsection{Participants}\label{participants}

This sample consisted of 101 participants consisting of men (20.8\%) and
women (79.2\%) aged between 18 and 72 years old (M = 35.24, SD = 10.65).
Recruited via undergraduates through Deakin University Australia.

\subsection{Data analysis}\label{data-analysis}

We used R (Version 3.5.1; R Core Team, 2018) and the R-packages
\emph{apaTables} (Version 2.0.5; Stanley, 2018), \emph{bindrcpp}
(Version 0.2.2; Müller, 2018), \emph{broom} (Version 0.5.0; Robinson \&
Hayes, 2018), \emph{car} (Version 3.0.2; Fox \& Weisberg, 2011; Fox,
Weisberg, \& Price, 2018), \emph{carData} (Version 3.0.2; Fox et al.,
2018), \emph{dplyr} (Version 0.7.6; Wickham, François, Henry, \& Müller,
2018), \emph{haven} (Version 1.1.2; Wickham \& Miller, 2018),
\emph{knitr} (Version 1.20; Xie, 2015), \emph{lmSupport} (Version
2.9.13; Curtin, 2018), \emph{MBESS} (Version 4.4.3; Kelley, 2018),
\emph{olsrr} (Version 0.5.1; Hebbali, 2018), \emph{papaja} (Version
0.1.0.9842; Aust \& Barth, 2018), \emph{psych} (Version 1.8.4; Revelle,
2018), and \emph{xtable} (Version 1.8.3; Dahl, Scott, Roosen, Magnusson,
\& Swinton, 2018) for all our analyses.

\textbf{\emph{Subjective Wellbeing}}. Wellbeing was assessed using the
Personality Wellbeing Index (PWI) to represent \enquote{life as a whole}
(R. Cummins et al., 2007; Group, 2013). The scale contains seven domains
of satisfaction that correspond to represent the overall \enquote{How
satisfied are you with life as whole?}. The domains are: standard of
living, health, achieving in life, relationships, safety,
community-connectedness, and future security. Participants indicated
their level of satisfaction with each domain using a 11-point Likert
scale ranging from 0 (\emph{No Satisfaction as all}) to 10
(\emph{Completely Satisfied}). Composite variables were then computed to
new variable and labelled as PWI. Cronbachs alpha has been shown to
yield small reliability estimates therefore, a more reliable coefficient
estimate, omega will be used (Dunn, Baguley, \& Brunsden, 2013; McNeish,
2017). Point estimates and confidence intervals for the current sample
were \(\omega\) = 0.74, 95\% CI {[}0.65, 0.83{]}.

\textbf{\emph{Personality}}. Personality was assessed using a 50-item
pool from the International Personality Item Pool (IPIP; Goldberg et al.
(2006)) which, provides estimates for the Five Factor Model. The five
factors and reliability estimates were as follows: Neuroticism
\(\omega\) = 0.86, 95\% CI {[}0.81, 0.90{]}, Extraversion \(\omega\) =
0.88, 95\% CI {[}0.84, 0.92{]}, Openness \(\omega\) = 0.72, 95\% CI
{[}0.60, 0.83{]}, Agreeableness \(\omega\) = 0.82, 95\% CI {[}0.76,
0.87{]} and Conscientiousness \(\omega\) = 0.77, 95\% CI {[}0.70,
0.83{]}.

\textbf{\emph{Affect}}. The core construct of SWB is that of
Homeostatically Protected Mood based on Russell's (2003) Core Affect (R.
Cummins, 2010; Davern et al., 2007). Participants were asked to describe
their feelings across three areas of life in general: How happy do you
feel? How content do you feel? And How alert do you feel? Each item was
scored on a Likert rated as 0 (\emph{Not at all}) to 10
(\emph{Extremely}). The three items were averaged and a single variable
was computed reflecting homeostatically protected mood (HPMood). The
reliability of the item for this sample was \(\omega\) = 0.86, 95\% CI
{[}0.79, 0.92{]}.

\textbf{\emph{Cognition}}. Items from a revised version of Multiple
Discrepancies Theory (MDT) was used to evaluate perceived gaps between
the current self and various standards of comparison (A. C. Michalos,
1985). This include: What one has and wants (\emph{self-wants}); what
relevant others have (\emph{self-other}); the best one has had in the
past (\emph{self-best}); what one expected to have 3 years ago
(\emph{selfprogress}); what one expects to have after 5 years
(\emph{self-future}); what one feels they deserve
(\emph{self-deserves}), and what one feels they need
(\emph{self-needs}). Items were averaged across each of the
discrepancies to indicate a gap between desired and actual life
circumstances. Scores closer to zero indicate less than desired
circumstances, scores around five reflect life circumstances close to
the current desired level and, scores close to 10 indicate actual
circumstances are better than desired. A single MDT variable was
computed and the reliability of the items was \(\omega\) = 0.78, 95\% CI
{[}0.72, 0.85{]}.

\subsection{Procedure}\label{procedure}

After obtaining ethical approval from Human Ethics Advisory Group of
Health (HEAG-H), participants were recruited through undergraduates via
a URL provided by Deakin University. The URL was then distributed
through email and social media networks. R Studio (Version 3.5.1; R Core
Team, 2018) was used to clean and analyze the data.

\section{Results}\label{results}

\subsection{Descriptives}\label{descriptives}

Means, standard deviations and correlations between variables are
presented below in Table 1. The wellbeing average was outside of the
normative range of scores ( \emph{M} = 68.80, \emph{SD} = 10.50) as
previous cohort studies shows the average for a sample should fall
between 73.43 - 76.43 points (R. Cummins, 2010). Neuroticism (\emph{r} =
-.74, \emph{p} \textless{}.01), HPMood (\emph{r} = .75, \emph{p}
\textless{}.01) and MDT ( \emph{r} = .60, \emph{p} \textless{}.01) were
all strongly correlated with PWI. Extraversion ( \emph{r} = .40,
\emph{p} \textless{}.01), Agreeableness ( \emph{r} = .27, \emph{p}
\textless{}.01) and Conscientiousness ( \emph{r} = .29, \emph{p}
\textless{}.01) were all moderately correlated with PWI.

Table 1

\label{tab:unnamed-chunk-1}

\emph{Correlation matrix of the key variables}

\begin{longtable}[]{@{}llllllllll@{}}
\toprule
& \(M\) & \(SD\) & 1 & 2 & 3 & 4 & 5 & 6 & 7\tabularnewline
\midrule
\endhead
PW\_Index & 68.80 & 10.50 & & & & & & &\tabularnewline
Neuroticism & 27.95 & 6.43 & -.74*** & & & & & &\tabularnewline
Extraversion & 31.76 & 6.78 & .40*** & -.31** & & & & &\tabularnewline
Openness & 36.78 & 4.94 & -.03 & .07 & .06 & & & &\tabularnewline
Agreeableness & 35.06 & 5.39 & .27** & -.28** & .03 & -.06 & &
&\tabularnewline
Conscientiousness & 33.96 & 5.05 & .29** & -.33*** & .23* & .10 & .23* &
&\tabularnewline
HPMood & 6.85 & 1.34 & .75*** & -.68*** & .34*** & -.04 & .34*** & .31**
&\tabularnewline
MDT & 6.02 & 1.24 & .60*** & -.50*** & .32** & .17+ & .13 & .21* &
.68***\tabularnewline
\bottomrule
\end{longtable}

\emph{Note.} Note. M and SD are used to represent mean and standard
deviation, respectively. Values in square brackets indicate the 95\%
confidence interval for each correlation. The confidence interval is a
plausible range of population correlations that could have caused the
sample correlation (Cumming, 2014). * indicates p \textless{} .05. **
indicates p \textless{} .01.

\subsection{Hierarchical Regression}\label{hierarchical-regression}

A three stage hierarchical multiple regression was then conducted with
PWI as the dependent variable. HPMood was entered at stage one to
control for affect. MDT was entered at stage two and the five
personality traits entered as a group at stage three. Regression
statistics are presented below with associated change statistics, zero
order correlations and regression weights.

The hierarchical multiple regression revealed that at stage one, HPMood
contributed significantly to the model and accounted for 57\% of the
variation in subjective wellbeing \(R^2 = .57\), 90\% CI \([0.45\),
\(0.67]\), \(F(1, 99) = 131.06\), \(p < .001\). Introducing MDT at stage
two explained 58\% of the variance \(R^2 = .58\), 90\% CI \([0.46\),
\(0.67]\), \(F(2, 98) = 68.24\), \(p < .001\) in SWB however, the R2
change was not significant \(t(98) = 1.70\), \(p = .091\). The final
step introduced the personality traits: Neuroticism, Extraversion,
Openness, Agreeableness and Conscientiousness and explained an
additional 10\% of the variance in SWB \(R^2 = .68\), 90\% CI \([0.56\),
\(0.75]\), \(F(7, 93) = 28.80\), \(p < .001\). The final step accounted
for 68\% of the variance in SWB. In the final adjusted model two out of
the seven predictors were statistically significant. Neuroticism had the
higher magnitude of effect \(b = -0.65\), 95\% CI \([-0.92\),
\(-0.38]\), \(t(93) = -4.84\), \(p < .001\) and the strongest unique
contribution to the overall model explaining 8\% of the variance in SWB.
HPMood \(b = 2.83\), 95\% CI \([1.27\), \(4.39]\), \(t(93) = 3.61\),
\(p = .001\) provided the next strongest effect size and uniquely
explained 4\% of the variance in SWB.

Table 2

\begin{table}[tbp]
\begin{center}
\begin{threeparttable}
\caption{\label{tab:unnamed-chunk-2}}
\small{
\begin{tabular}{llll}
\toprule
Variables & \multicolumn{1}{c}{Step1} & \multicolumn{1}{c}{Step2} & \multicolumn{1}{c}{Step3}\\
\midrule
Intercept & $28.25$ $[21.10$, $35.41]$ & $26.10$ $[18.58$, $33.62]$ & $56.10$ $[35.84$, $76.36]$\\
HPMood & $5.92$ $[4.89$, $6.95]$ & $5.10$ $[3.71$, $6.50]$ & $2.83$ $[1.27$, $4.39]$\\
MDT &  & $1.29$ $[-0.21$, $2.78]$ & $0.98$ $[-0.43$, $2.39]$\\
Agreeableness &  &  & $0.03$ $[-0.22$, $0.27]$\\
Conscientiousness &  &  & $-0.01$ $[-0.27$, $0.25]$\\
Extraversion &  &  & $0.18$ $[-0.02$, $0.37]$\\
Neuroticism &  &  & $-0.65$ $[-0.92$, $-0.38]$\\
Openness &  &  & $-0.02$ $[-0.27$, $0.24]$\\
$R^2$ [90\% CI] & $.57$ $[0.45$, $0.67]$ & $.58$ $[0.46$, $0.67]$ & $.68$ $[0.56$, $0.75]$\\
$F$ & 131.06 & 68.24 & 28.80\\
$df_1$ & 2 & 3 & 8\\
$df_2$ & 99 & 98 & 93\\
$p$ & < .001 & < .001 & < .001\\
$\mathrm{AIC}$ & 681.46 & 680.51 & 662.18\\
$\mathrm{BIC}$ & 689.30 & 690.97 & 685.71\\
$\Delta R^2$ &  & $.01$ & $.10$\\
$F$ &  & 3.65 & 6.02\\
$df_1$ &  & 1 & 5\\
$df_2$ &  & 99 & 99\\
$p$ &  & .059 & < .001\\
$\Delta \mathrm{AIC}$ &  & -0.95 & -18.33\\
$\Delta \mathrm{BIC}$ &  & 1.66 & -5.25\\
\bottomrule
\addlinespace
\end{tabular}
}
\begin{tablenotes}[para]
\normalsize{\textit{Note.} Note. A significant b-weight indicates the beta-weight and semi-partial correlation are also significant. b represents unstandardized regression weights. beta indicates the standardized regression weights. sr2 represents the semi-partial correlation squared. r represents the zero-order correlation. LL and UL indicate the lower and upper limits of a confidence interval, respectively.
* indicates p < .05. ** indicates p < .01.
}
\end{tablenotes}
\end{threeparttable}
\end{center}
\end{table}

Further exploratory model was then conducted excluding non-significant
coefficients. A multiple regression analysis showed that HPMood and
Neuroticism significantly predicted SWB (\(R^2 = .66\), 90\% CI
\([0.56\), \(0.74]\), \(F(2, 98) = 97.02\), \(p < .001\)) and accounted
for 66\% of the variance in SWB. HPMood proved to have the strongest
effect (\(\beta = 3.65\), 95\% CI \([2.40\), \(4.90]\),
\(t(98) = 5.80\), \(p < .001\)) accounting for 12\% of the unique
variance in SWB followed by Neuroticism (\(\beta = -.69\), 95\% CI
\([-.95\), \(-.43]\), \(t(98) = -5.26\), \(p < .001\)) which contributed
9\% of the unique variance.

Table 3

\begin{table}[tbp]
\begin{center}
\begin{threeparttable}
\caption{\label{tab:unnamed-chunk-3}}
\small{
\begin{tabular}{lllll}
\toprule
Predictor & \multicolumn{1}{c}{$b$} & \multicolumn{1}{c}{95\% CI} & \multicolumn{1}{c}{$t(98)$} & \multicolumn{1}{c}{$p$}\\
\midrule
Intercept & 63.07 & $[48.48$, $77.66]$ & 8.58 & < .001\\
HPMood & 3.65 & $[2.40$, $4.90]$ & 5.80 & < .001\\
Neuroticism & -0.69 & $[-0.95$, $-0.43]$ & -5.26 & < .001\\
\bottomrule
\addlinespace
\end{tabular}
}
\begin{tablenotes}[para]
\normalsize{\textit{Note.} This is a note section}
\end{tablenotes}
\end{threeparttable}
\end{center}
\end{table}

\subsection{Discussion}\label{discussion}

Despite the position of Diener et al., (2018) that the cognitive
component plays the larger role in wellbeing, previous research has
shown that an individual's subjective wellbeing is driven by affect and
managed through a neurophysiological process (R. Cummins, 2010). Results
from similar designs either used unreliable personality measures or no
reliability measures were reported. This study aimed to re-examine the
components of affect, cognition and personality on an individual's
subjective wellbeing and examine the normative range of wellbeing
scores.

Results from hierarchical regression indicated that affect was the
dominant component, consistent with that of Davern et al. (2007), Tomyn
\& Cummins (2011), and Blore et al. (2011), accounting for 57\% of the
variation in SWB. The cognition component explained an additional 10\%
of the variance however, the improvement in the model was not
significant. This would explain the significance of when the personality
component was entered, accounting for the additional 10\% of the
variance in SWB. Indicating that the personality trait of neuroticism
takes control of the structure of wellbeing when homeostasis is
disrupted, not cognition, inconsistent with Davern et al.'s (2007)s
hypothesis.

An unexpected result was that of the samples wellbeing score. The
samples mean score (68.80\%) was outside the normative range and
therefor, evidence there was instability to the homeostatic process
(Capic et al., 2018; R. Cummins, 2010). This becomes evident as the
model is processed through after controlling for affect. Based on these
results, further exploratory analysis was then conducted with affect and
neuroticism as the predictor variables. Both affect, and neuroticism
were significant and explained 66\% of the variance in SWB. Furthermore,
affect contributed the dominant effect size.

A key explanation for these findings is that when SWB is disrupted, the
cognitive component drops out and neuroticism becomes the mediating role
on wellbeing (Fadda \& Scalas, 2016). Inconsistent with Davern et al.,
(2007) hypothesis that when there is a disruption to homeostatic
process, cognition will take control and explain more variance in SWB
than affect. The importance of this is that personality traits of
Neuroticism, or Extraversion play the key role in determining how an
individual is likely to cope (Costa \& McCrae, 1980). This is consistent
with the idea that under challenging circumstances SWB is lower
(Richardson et al., 2016). Therefore, the individual is more likely to
rely on emotion rather than cognition when experiencing challenging
experiences.

Under normal conditions, the cognitive-affective model is the preferred
structure as demonstrated with Davern et al., (2007). If the sample size
is reflective of the adult population, the model should operate under
the homeostasis presumption. However, if the sample is likely to be
youth or evidence of psychopathology, then a purely affective model
would explain their SWB, guided by their level of neuroticism. This has
two implications firstly, the cognitive-affective model is useful in
guiding public policy, such as broad health interventions and secondly,
the affective model has clinical applications. Assessing an individual's
wellbeing and personality could highlight how they are likely to cope
and therefore, relevant interventions can then be applied.

In light of these findings there were some limitations. Firstly, gender
distribution was heavily weighted toward females and may have attenuated
the regression line. To date, gender differences in SWB have yet to be
fully explored therefore, future designs could implement a hierarchical
process by accounting for affect and entering gender separately.
Secondly, both this and previous studies have used a cross sectional
methodology. A longitudinal approach may highlight the long-term
stability of homeostasis by addressing the variation in traits.

Overall, present findings suggest that affect explains most of the
variance in SWB when wellbeing is in a state of challenge. When the
wellbeing is outside the range of normality, trait neuroticism overrides
cognition and takes control. Further research is needed in order to
address the limitations in this study, gender differences have not been
fully examined and personality variations may influence affect over
time. Future research could consider exploring the relative gender
differences in wellbeing and adopt a longitudinal design.

\newpage

\section{References}\label{references}

\begingroup
\setlength{\parindent}{-0.5in} \setlength{\leftskip}{0.5in}

\hypertarget{refs}{}
\hypertarget{ref-R-papaja}{}
Aust, F., \& Barth, M. (2018). \emph{papaja: Create APA manuscripts with
R Markdown}. Retrieved from \url{https://github.com/crsh/papaja}

\hypertarget{ref-RN420}{}
Blore, J. D., Stokes, M. A., Mellor, D., Firth, L., \& Cummins, R. A.
(2011). Comparing multiple discrepancies theory to affective models of
subjective wellbeing. \emph{Social Indicators Research}, \emph{100}(1),
1--16. Journal Article.
doi:\href{https://doi.org/10.1007/s11205-010-9599-2}{10.1007/s11205-010-9599-2}

\hypertarget{ref-RN382}{}
Capic, T., Li, N., \& Cummins, R. A. (2018). Confirmation of subjective
wellbeing set-points: Foundational for subjective social indicators.
\emph{Social Indicators Research}, \emph{137}(1), 1--28. Journal
Article.
doi:\href{https://doi.org/10.1007/s11205-017-1585-5}{10.1007/s11205-017-1585-5}

\hypertarget{ref-RN421}{}
Cummins, R. (2010). Subjective wellbeing, homeostatically protected mood
and depression: A synthesis. \emph{Journal of Happiness Studies},
\emph{11}(1), 1--17. Journal Article.
doi:\href{https://doi.org/10.1007/s10902-009-9167-0}{10.1007/s10902-009-9167-0}

\hypertarget{ref-RN546}{}
Cummins, R., Hughes, J., Tomyn, A., Gibson, A., Woerner, J., \& Lai, L.
(2007). \emph{The wellbeing of australians: Carer health and wellbeing}.
Book, Geelong, Vic.: Deakin University. Retrieved from
\url{http://ezproxy.deakin.edu.au/login?url=http://search.ebscohost.com/login.aspx?direct=true\&db=edsafs\&AN=edsafs.a59030\&authtype=sso\&custid=deakin\&site=eds-live\&scope=site}

\hypertarget{ref-RN305}{}
Cummins, R., Li, N., Wooden, M., \& Stokes, M. (2014). A demonstration
of set-points for subjective wellbeing. \emph{Journal of Happiness
Studies}, \emph{15}(1), 183--206. Journal Article.
doi:\href{https://doi.org/10.1007/s10902-013-9444-9}{10.1007/s10902-013-9444-9}

\hypertarget{ref-R-lmSupport}{}
Curtin, J. (2018). \emph{LmSupport: Support for linear models}.
Retrieved from \url{https://CRAN.R-project.org/package=lmSupport}

\hypertarget{ref-R-xtable}{}
Dahl, D. B., Scott, D., Roosen, C., Magnusson, A., \& Swinton, J.
(2018). \emph{Xtable: Export tables to latex or html}. Retrieved from
\url{https://CRAN.R-project.org/package=xtable}

\hypertarget{ref-RN417}{}
Davern, M. T., Cummins, R. A., \& Stokes, M. A. (2007). Subjective
wellbeing as an affective-cognitive construct. \emph{Journal of
Happiness Studies}, \emph{8}(4), 429--449. Journal Article.
doi:\href{https://doi.org/10.1007/s10902-007-9066-1}{10.1007/s10902-007-9066-1}

\hypertarget{ref-RN531}{}
Dunn, T. J., Baguley, T., \& Brunsden, V. (2013). From alpha to omega: A
practical solution to the pervasive problem of internal consistency
estimation. \emph{British Journal of Psychology}, \emph{105}(3),
399--412. Journal Article.
doi:\href{https://doi.org/10.1111/bjop.12046}{10.1111/bjop.12046}

\hypertarget{ref-R-car}{}
Fox, J., \& Weisberg, S. (2011). \emph{An R companion to applied
regression} (Second.). Thousand Oaks CA: Sage. Retrieved from
\url{http://socserv.socsci.mcmaster.ca/jfox/Books/Companion}

\hypertarget{ref-R-carData}{}
Fox, J., Weisberg, S., \& Price, B. (2018). \emph{CarData: Companion to
applied regression data sets}. Retrieved from
\url{https://CRAN.R-project.org/package=carData}

\hypertarget{ref-RN566}{}
Goldberg, L. R., Johnson, J. A., Eber, H. W., Hogan, R., Ashton, M. C.,
Cloninger, C. R., \& Gough, H. G. (2006). The international personality
item pool and the future of public-domain personality measures.
\emph{Journal of Research in Personality}, \emph{40}(1), 84--96. Journal
Article.
doi:\href{https://doi.org/10.1016/j.jrp.2005.08.007}{10.1016/j.jrp.2005.08.007}

\hypertarget{ref-RN534}{}
Gomez, V., Allemand, M., \& Grob, A. (2012). Neuroticism, extraversion,
goals, and subjective well-being: Exploring the relations in young,
middle-aged, and older adults. \emph{Journal of Research in
Personality}, \emph{46}(3), 317--325. Journal Article.
doi:\href{https://doi.org/https://doi.org/10.1016/j.jrp.2012.03.001}{https://doi.org/10.1016/j.jrp.2012.03.001}

\hypertarget{ref-RN567}{}
Group, I. W. (2013). Personal wellbeing index. \emph{5th Edition.
Melbourne: Australian}. Journal Article. Retrieved from
\url{http://www.deakin.edu.au/research/acqol/instruments/wellbeing-index/index.php)}

\hypertarget{ref-RN558}{}
Harris, K., English, T., Harms, P. D., Gross, J. J., Jackson, J. J., \&
Back, M. (2017). Why are extraverts more satisfied? Personality, social
experiences, and subjective well-being in college. \emph{European
Journal of Personality}, \emph{31}(2), 170--186. Journal Article.
doi:\href{https://doi.org/10.1002/per.2101}{10.1002/per.2101}

\hypertarget{ref-R-olsrr}{}
Hebbali, A. (2018). \emph{Olsrr: Tools for building ols regression
models}. Retrieved from \url{https://CRAN.R-project.org/package=olsrr}

\hypertarget{ref-RN426}{}
Jovanovic, V. (2011). Personality and subjective well-being: One
neglected model of personality and two forgotten aspects of subjective
well-being. \emph{Personality and Individual Differences}, \emph{50}(5),
631--635. Journal Article.
doi:\href{https://doi.org/https://doi.org/10.1016/j.paid.2010.12.008}{https://doi.org/10.1016/j.paid.2010.12.008}

\hypertarget{ref-R-MBESS}{}
Kelley, K. (2018). \emph{MBESS: The mbess r package}. Retrieved from
\url{https://CRAN.R-project.org/package=MBESS}

\hypertarget{ref-RN549}{}
Lucas, R. E. (2008). Personality and subjective well-being. In \emph{The
science of subjective well-being.} (pp. 171--194). Book Section, New
York, NY, US: Guilford Press.

\hypertarget{ref-RN532}{}
McNeish, D. (2017). \emph{Thanks coefficient alpha, we'll take it from
here}. Book.
doi:\href{https://doi.org/10.1037/met0000144}{10.1037/met0000144}

\hypertarget{ref-RN300}{}
Michalos, A. (2017). \emph{Multiple discrepancies theory (mdt)} (pp.
39--95). Book.
doi:\href{https://doi.org/10.1007/978-3-319-51149-8_3}{10.1007/978-3-319-51149-8\_3}

\hypertarget{ref-RN569}{}
Michalos, A. C. (1985). Multiple discrepancies theory (mdt).
\emph{Social Indicators Research}, \emph{16}(4), 347--413. Journal
Article.
doi:\href{https://doi.org/10.1007/BF00333288}{10.1007/BF00333288}

\hypertarget{ref-R-bindrcpp}{}
Müller, K. (2018). \emph{Bindrcpp: An 'rcpp' interface to active
bindings}. Retrieved from
\url{https://CRAN.R-project.org/package=bindrcpp}

\hypertarget{ref-RN301}{}
Pollock, N., Noser, A., Holden, C., \& Zeigler-Hill, V. (2016). Do
orientations to happiness mediate the associations between personality
traits and subjective well-being? \emph{Journal of Happiness Studies},
\emph{17}(2), 713--729. Journal Article.
doi:\href{https://doi.org/10.1007/s10902-015-9617-9}{10.1007/s10902-015-9617-9}

\hypertarget{ref-R-base}{}
R Core Team. (2018). \emph{R: A language and environment for statistical
computing}. Vienna, Austria: R Foundation for Statistical Computing.
Retrieved from \url{https://www.R-project.org/}

\hypertarget{ref-R-psych}{}
Revelle, W. (2018). \emph{Psych: Procedures for psychological,
psychometric, and personality research}. Evanston, Illinois:
Northwestern University. Retrieved from
\url{https://CRAN.R-project.org/package=psych}

\hypertarget{ref-RN383}{}
Richardson, B., Fuller Tyszkiewicz, M., Tomyn, A., \& Cummins, R.
(2016). The psychometric equivalence of the personal wellbeing index for
normally functioning and homeostatically defeated australian adults.
\emph{Journal of Happiness Studies}, \emph{17}(2), 627--641. Journal
Article.
doi:\href{https://doi.org/10.1007/s10902-015-9613-0}{10.1007/s10902-015-9613-0}

\hypertarget{ref-R-broom}{}
Robinson, D., \& Hayes, A. (2018). \emph{Broom: Convert statistical
analysis objects into tidy tibbles}. Retrieved from
\url{https://CRAN.R-project.org/package=broom}

\hypertarget{ref-RN572}{}
Rosellini, A. J., \& Brown, T. A. (2011). The neo five-factor inventory:
Latent structure and relationships with dimensions of anxiety and
depressive disorders in a large clinical sample. \emph{Assessment},
\emph{18}(1), 27--38. Journal Article.
doi:\href{https://doi.org/10.1177/1073191110382848}{10.1177/1073191110382848}

\hypertarget{ref-R-apaTables}{}
Stanley, D. (2018). \emph{ApaTables: Create american psychological
association (apa) style tables}. Retrieved from
\url{https://CRAN.R-project.org/package=apaTables}

\hypertarget{ref-RN418}{}
Tomyn, A., \& Cummins, R. (2011). Subjective wellbeing and
homeostatically protected mood: Theory validation with adolescents.
\emph{Journal of Happiness Studies}, \emph{12}(5), 897--914. Journal
Article.
doi:\href{https://doi.org/10.1007/s10902-010-9235-5}{10.1007/s10902-010-9235-5}

\hypertarget{ref-R-haven}{}
Wickham, H., \& Miller, E. (2018). \emph{Haven: Import and export
'spss', 'stata' and 'sas' files}. Retrieved from
\url{https://CRAN.R-project.org/package=haven}

\hypertarget{ref-R-dplyr}{}
Wickham, H., François, R., Henry, L., \& Müller, K. (2018). \emph{Dplyr:
A grammar of data manipulation}. Retrieved from
\url{https://CRAN.R-project.org/package=dplyr}

\hypertarget{ref-R-knitr}{}
Xie, Y. (2015). \emph{Dynamic documents with R and knitr} (2nd ed.).
Boca Raton, Florida: Chapman; Hall/CRC. Retrieved from
\url{https://yihui.name/knitr/}

\endgroup

\clearpage

\renewcommand{\listfigurename}{Figure captions}

\listoffigures


\end{document}
