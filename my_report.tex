\documentclass[man,floatsintext]{apa6}
\usepackage{lmodern}
\usepackage{amssymb,amsmath}
\usepackage{ifxetex,ifluatex}
\usepackage{fixltx2e} % provides \textsubscript
\ifnum 0\ifxetex 1\fi\ifluatex 1\fi=0 % if pdftex
  \usepackage[T1]{fontenc}
  \usepackage[utf8]{inputenc}
\else % if luatex or xelatex
  \ifxetex
    \usepackage{mathspec}
  \else
    \usepackage{fontspec}
  \fi
  \defaultfontfeatures{Ligatures=TeX,Scale=MatchLowercase}
\fi
% use upquote if available, for straight quotes in verbatim environments
\IfFileExists{upquote.sty}{\usepackage{upquote}}{}
% use microtype if available
\IfFileExists{microtype.sty}{%
\usepackage{microtype}
\UseMicrotypeSet[protrusion]{basicmath} % disable protrusion for tt fonts
}{}
\usepackage{hyperref}
\hypersetup{unicode=true,
            pdftitle={Competing Models of SWB},
            pdfauthor={Aaron L. Willcox},
            pdfkeywords={keywords},
            pdfborder={0 0 0},
            breaklinks=true}
\urlstyle{same}  % don't use monospace font for urls
\usepackage{longtable,booktabs}
\usepackage{graphicx,grffile}
\makeatletter
\def\maxwidth{\ifdim\Gin@nat@width>\linewidth\linewidth\else\Gin@nat@width\fi}
\def\maxheight{\ifdim\Gin@nat@height>\textheight\textheight\else\Gin@nat@height\fi}
\makeatother
% Scale images if necessary, so that they will not overflow the page
% margins by default, and it is still possible to overwrite the defaults
% using explicit options in \includegraphics[width, height, ...]{}
\setkeys{Gin}{width=\maxwidth,height=\maxheight,keepaspectratio}
\IfFileExists{parskip.sty}{%
\usepackage{parskip}
}{% else
\setlength{\parindent}{0pt}
\setlength{\parskip}{6pt plus 2pt minus 1pt}
}
\setlength{\emergencystretch}{3em}  % prevent overfull lines
\providecommand{\tightlist}{%
  \setlength{\itemsep}{0pt}\setlength{\parskip}{0pt}}
\setcounter{secnumdepth}{0}
% Redefines (sub)paragraphs to behave more like sections
\ifx\paragraph\undefined\else
\let\oldparagraph\paragraph
\renewcommand{\paragraph}[1]{\oldparagraph{#1}\mbox{}}
\fi
\ifx\subparagraph\undefined\else
\let\oldsubparagraph\subparagraph
\renewcommand{\subparagraph}[1]{\oldsubparagraph{#1}\mbox{}}
\fi

%%% Use protect on footnotes to avoid problems with footnotes in titles
\let\rmarkdownfootnote\footnote%
\def\footnote{\protect\rmarkdownfootnote}


  \title{Competing Models of SWB}
    \author{Aaron L. Willcox\textsuperscript{1}}
    \date{}
  
\shorttitle{Models of SWB}
\affiliation{
\vspace{0.5cm}
\textsuperscript{1} Deakin University}
\keywords{keywords\newline\indent Word count: X}
\usepackage{csquotes}
\usepackage{upgreek}
\captionsetup{font=singlespacing,justification=justified}

\usepackage{longtable}
\usepackage{lscape}
\usepackage{multirow}
\usepackage{tabularx}
\usepackage[flushleft]{threeparttable}
\usepackage{threeparttablex}

\newenvironment{lltable}{\begin{landscape}\begin{center}\begin{ThreePartTable}}{\end{ThreePartTable}\end{center}\end{landscape}}

\makeatletter
\newcommand\LastLTentrywidth{1em}
\newlength\longtablewidth
\setlength{\longtablewidth}{1in}
\newcommand{\getlongtablewidth}{\begingroup \ifcsname LT@\roman{LT@tables}\endcsname \global\longtablewidth=0pt \renewcommand{\LT@entry}[2]{\global\advance\longtablewidth by ##2\relax\gdef\LastLTentrywidth{##2}}\@nameuse{LT@\roman{LT@tables}} \fi \endgroup}



\authornote{

Correspondence concerning this article should be addressed to Aaron L.
Willcox, Postal address. E-mail:
\href{mailto:my@email.com}{\nolinkurl{my@email.com}}}

\abstract{
One or two sentences providing a \textbf{basic introduction} to the
field, comprehensible to a scientist in any discipline.

Two to three sentences of \textbf{more detailed background},
comprehensible to scientists in related disciplines.

One sentence clearly stating the \textbf{general problem} being
addressed by this particular study.

One sentence summarizing the main result (with the words ``\textbf{here
we show}'' or their equivalent).

Two or three sentences explaining what the \textbf{main result} reveals
in direct comparison to what was thought to be the case previously, or
how the main result adds to previous knowledge.

One or two sentences to put the results into a more \textbf{general
context}.

Two or three sentences to provide a \textbf{broader perspective},
readily comprehensible to a scientist in any discipline.


}

\begin{document}
\maketitle

\section{Introduction}\label{introduction}

Subjective Wellbeing is the overall evaluation of feeling satisfied
absed on cognitive, emotional domains. However, the contributions as to
how much of each domain is still under much debate. The Australian Unity
Wellbeing Index provides a snapshot of the overwellbeing of Australians
each year. This study has a relateively robust score and has deomstrated
that, on average, the population PWI average is 75 with a 3.0 percentage
points difference. What this suggests is that a persons wellbeing has a
set-point as to where their sense of happiness is and therefore can be
used to evaluate wellbeing.

\section{Methods}\label{methods}

We report how we determined our sample size, all data exclusions (if
any), all manipulations, and all measures in the study.

Measures of Employment history were also recorded but were not included
in the analyis.

\subsection{Participants}\label{participants}

\subsection{Material}\label{material}

\subsection{Procedure}\label{procedure}

\subsection{Data analysis}\label{data-analysis}

We used R (Version 3.5.1; R Core Team, 2018) and the R-packages
\emph{apaTables} (Version 2.0.5; Stanley, 2018), \emph{bindrcpp}
(Version 0.2.2; Müller, 2018), \emph{broom} (Version 0.5.0; Robinson \&
Hayes, 2018), \emph{car} (Version 3.0.2; Fox \& Weisberg, 2011; Fox,
Weisberg, \& Price, 2018), \emph{carData} (Version 3.0.1; Fox et al.,
2018), \emph{dplyr} (Version 0.7.6; Wickham, François, Henry, \& Müller,
2018), \emph{haven} (Version 1.1.2; Wickham \& Miller, 2018),
\emph{knitr} (Version 1.20; Xie, 2015), \emph{lmSupport} (Version
2.9.13; Curtin, 2018), \emph{MBESS} (Version 4.4.3; Kelley, 2018),
\emph{olsrr} (Version 0.5.1; Hebbali, 2018), \emph{papaja} (Version
0.1.0.9842; Aust \& Barth, 2018), \emph{psych} (Version 1.8.4; Revelle,
2018), and \emph{xtable} (Version 1.8.3; Dahl, Scott, Roosen, Magnusson,
\& Swinton, 2018) for all our analyses.

\section{Results}\label{results}

I need to print and omega symbol \(\omega\) and I hope that this worked
maybe if i try this one or perhaps this one

\section{Method}\label{method}

\textbf{\emph{Subjective Wellbeing}}. Wellbeing was assessed using the
Personality Wellbeing Index (PWI) to represent \enquote{life as a whole}
\{Cummins, 2003 \#547\} \{Group, 2013 \#567\}. The scale contains seven
domains of satisfaction that correspond to represent the overall
\enquote{How satisfied are you with life as whole?}. The domains are:
standard of living, health, achieving in life, relationships, safety,
community-connectedness, and future security. Participants indicated
their level of satisfaction with each domain using a 11-point Likert
scale ranging from 0 (No Satisfaction as all) to 10 (Completely
Satisfied). Composite variables were then computed to new variable and
labelled as PWI. Cronbachs alpha has been shown to yield small
reliability estimates therefore, a more reliable coefficient estimate,
omega will be used \{McNeish, 2017 \#532\}\{Dunn, 2013 \#531\}. Point
estimates and confidence intervals for the current sample were
\(\omega\) = 0.74, 95\% CI {[}0.65, 0.83{]}.

\textbf{\emph{Personality}}. Personality was assessed using a 50-item
pool from the International Personality Item Pool (IPIP;\{Goldberg, 2006
\#566\} ) which, provides estimates for the Five Factor Model. The five
factors and reliability estimates were as follows: Neuroticism
\(\omega\) = 0.86, 95\% CI {[}0.81, 0.90{]}, Extraversion \(\omega\) =
0.88, 95\% CI {[}0.84, 0.92{]}, Openness \(\omega\) = 0.72, 95\% CI
{[}0.60, 0.83{]}, Agreeableness \(\omega\) = 0.82, 95\% CI {[}0.76,
0.87{]} and Conscientiousness \(\omega\) = 0.77, 95\% CI {[}0.70,
0.83{]}.

\textbf{\emph{Affect}}. The core construct of SWB is that of
Homeostatically Protected Mood based on Russell's (2003) Core Affect
\{Cummins, 2010 \#421\} \{Davern, 2007 \#417\}. Participants were asked
to describe their feelings across three areas of life in general: How
happy do you feel? How content do you feel? And How alert do you feel?
Each item was scored on a Likert rated as 0 (Not at all) to 10
(Extremely). The three items were averaged and a single variable was
computed reflecting homeostatically protected mood (HPMood). The
reliability of the item for this sample was \(\omega\) = 0.86, 95\% CI
{[}0.79, 0.92{]}.

\textbf{\emph{Cognition}}. Items from a revised version of Multiple
Discrepancies Theory (MDT) was used to evaluate perceived gaps between
the current self and various standards of comparison \{Michalos, 1985
\#569\}. This include: What one has and wants (self-wants); what
relevant others have (self-other); the best one has had in the past
(self-best); what one expected to have 3 years ago (selfprogress); what
one expects to have after 5 years (self-future); what one feels they
deserve (self-deserves), and what one feels they need (self-needs).
Items were averaged across each of the discrepancies to indicate a gap
between desired and actual life circumstances. Scores closer to zero
indicate less than desired circumstances, scores around five reflect
life circumstances close to the current desired level and, scores close
to 10 indicate actual circumstances are better than desired. A single
MDT variable was computed and the reliability of the items was
\(\omega\) = 0.78, 95\% CI {[}0.72, 0.85{]}.

\section{Results}\label{results-1}

Prior to conducting a hierarchical multiple regression, relevant
assumptions of the model were tested. Standardized residuals and qqplots
indicated that normality, linearity and multicollinearity were
acceptable. However, an examination of Cooks Distance indicated several
influential cases which were then removed to prevent extreme leveraging
of the model \{Cook, 1982 \#570\}. Means, standard deviations and
correlations between variables are presented below in Table 1. The
wellbeing average was outside of the normative range of scores (M =
68.80, SD = 10.50) as previous cohort studies shows the average for a
sample should fall between 73.43 - 76.43 points \{Cummins, 2010 \#421\}.
Neuroticism (r = -.74, p \textless{}.01), HPMood (r = .75, p
\textless{}.01) and MDT (r = .60, p \textless{}.01) were all strongly
correlated with PWI. Extraversion (r= .40, p \textless{}.01),
Agreeableness (r = .27, p \textless{}.01) and Conscientiousness (r =
.29, p \textless{}.01) were all moderately correlated with PWI.

A three stage hierarchical multiple regression was then conducted with
PWI as the dependent variable. HPMood was entered at stage one to
control for affect. MDT was entered at stage two and the five
personality traits entered as a group at stage three. Regression
statistics are presented below with associated change statistics, zero
order correlations and regression weights.

The hierarchical multiple regression revealed that at stage one, HPMood
contributed significantly to the model, and accounted for 57\% of the
variation in subjective wellbeing . Introducing MDT at stage two
explained 58\% of the variance in SWB however, the R2 change was not
significant\(t(98) = 1.70\), \(p = .091\). The final step introduced the
personality traits: Neuroticism, Extraversion, Openness, Agreeableness
and Conscientiousness and explained an additional 10\% of the variance
in SWB . The final step accounted for 66\% of the variance in SWB. In
the final adjusted model two out of the seven predictors were
statistically significant. Neuroticism had the higher magnitude of
effect \(b = 2.83\), 95\% CI \([1.27\), \(4.39]\), \(t(93) = 3.61\),
\(p = .001\) and the strongest unique contribution to the overall model
explaining 8\% of the variance in SWB. HPMood provided the next
strongest effect size and uniquely explained 4\% of the variance in SWB.

\begin{table}[tbp]
\begin{center}
\begin{threeparttable}
\caption{\label{tab:unnamed-chunk-1}}
\small{
\begin{tabular}{llll}
\toprule
Variables & \multicolumn{1}{c}{Step1} & \multicolumn{1}{c}{Step2} & \multicolumn{1}{c}{Step3}\\
\midrule
Intercept & $28.25$ $[21.10$, $35.41]$ & $26.10$ $[18.58$, $33.62]$ & $56.10$ $[35.84$, $76.36]$\\
HPMood & $5.92$ $[4.89$, $6.95]$ & $5.10$ $[3.71$, $6.50]$ & $2.83$ $[1.27$, $4.39]$\\
MDT &  & $1.29$ $[-0.21$, $2.78]$ & $0.98$ $[-0.43$, $2.39]$\\
Agreeableness &  &  & $0.03$ $[-0.22$, $0.27]$\\
Conscientiousness &  &  & $-0.01$ $[-0.27$, $0.25]$\\
Extraversion &  &  & $0.18$ $[-0.02$, $0.37]$\\
Neuroticism &  &  & $-0.65$ $[-0.92$, $-0.38]$\\
Openness &  &  & $-0.02$ $[-0.27$, $0.24]$\\
$R^2$ [90\% CI] & $.57$ $[0.45$, $0.67]$ & $.58$ $[0.46$, $0.67]$ & $.68$ $[0.56$, $0.75]$\\
$F$ & 131.06 & 68.24 & 28.80\\
$df_1$ & 2 & 3 & 8\\
$df_2$ & 99 & 98 & 93\\
$p$ & < .001 & < .001 & < .001\\
$\mathrm{AIC}$ & 681.46 & 680.51 & 662.18\\
$\mathrm{BIC}$ & 689.30 & 690.97 & 685.71\\
$\Delta R^2$ &  & $.01$ & $.10$\\
$F$ &  & 3.65 & 6.02\\
$df_1$ &  & 1 & 5\\
$df_2$ &  & 99 & 99\\
$p$ &  & .059 & < .001\\
$\Delta \mathrm{AIC}$ &  & -0.95 & -18.33\\
$\Delta \mathrm{BIC}$ &  & 1.66 & -5.25\\
\bottomrule
\addlinespace
\end{tabular}
}
\begin{tablenotes}[para]
\normalsize{\textit{Note.} This is a note section}
\end{tablenotes}
\end{threeparttable}
\end{center}
\end{table}

Then on the next page we have here a loverly correlation table for me to
check it and see if the covaiates match with the regression model

\label{tab:unnamed-chunk-2}

\emph{Correlation matrix of the main variables}

\begin{longtable}[]{@{}llllllllll@{}}
\toprule
& \(M\) & \(SD\) & 1 & 2 & 3 & 4 & 5 & 6 & 7\tabularnewline
\midrule
\endhead
PW\_Index & 68.80 & 10.50 & & & & & & &\tabularnewline
Neuroticism & 27.95 & 6.43 & -.74*** & & & & & &\tabularnewline
Extraversion & 31.76 & 6.78 & .40*** & -.31** & & & & &\tabularnewline
Openness & 36.78 & 4.94 & -.03 & .07 & .06 & & & &\tabularnewline
Agreeableness & 35.06 & 5.39 & .27** & -.28** & .03 & -.06 & &
&\tabularnewline
Conscientiousness & 33.96 & 5.05 & .29** & -.33*** & .23* & .10 & .23* &
&\tabularnewline
HPMood & 6.85 & 1.34 & .75*** & -.68*** & .34*** & -.04 & .34*** & .31**
&\tabularnewline
MDT & 6.02 & 1.24 & .60*** & -.50*** & .32** & .17+ & .13 & .21* &
.68***\tabularnewline
\bottomrule
\end{longtable}

\emph{Note.} Note. M and SD are used to represent mean and standard
deviation, respectively. Values in square brackets indicate the 95\%
confidence interval for each correlation. The confidence interval is a
plausible range of population correlations that could have caused the
sample correlation (Cumming, 2014). * indicates p \textless{} .05. **
indicates p \textless{} .01.

\section{Discussion}\label{discussion}

\newpage

\section{References}\label{references}

\begingroup
\setlength{\parindent}{-0.5in} \setlength{\leftskip}{0.5in}

\hypertarget{refs}{}
\hypertarget{ref-R-papaja}{}
Aust, F., \& Barth, M. (2018). \emph{papaja: Create APA manuscripts with
R Markdown}. Retrieved from \url{https://github.com/crsh/papaja}

\hypertarget{ref-R-lmSupport}{}
Curtin, J. (2018). \emph{LmSupport: Support for linear models}.
Retrieved from \url{https://CRAN.R-project.org/package=lmSupport}

\hypertarget{ref-R-xtable}{}
Dahl, D. B., Scott, D., Roosen, C., Magnusson, A., \& Swinton, J.
(2018). \emph{Xtable: Export tables to latex or html}. Retrieved from
\url{https://CRAN.R-project.org/package=xtable}

\hypertarget{ref-R-car}{}
Fox, J., \& Weisberg, S. (2011). \emph{An R companion to applied
regression} (Second.). Thousand Oaks CA: Sage. Retrieved from
\url{http://socserv.socsci.mcmaster.ca/jfox/Books/Companion}

\hypertarget{ref-R-carData}{}
Fox, J., Weisberg, S., \& Price, B. (2018). \emph{CarData: Companion to
applied regression data sets}. Retrieved from
\url{https://CRAN.R-project.org/package=carData}

\hypertarget{ref-R-olsrr}{}
Hebbali, A. (2018). \emph{Olsrr: Tools for building ols regression
models}. Retrieved from \url{https://CRAN.R-project.org/package=olsrr}

\hypertarget{ref-R-MBESS}{}
Kelley, K. (2018). \emph{MBESS: The mbess r package}. Retrieved from
\url{https://CRAN.R-project.org/package=MBESS}

\hypertarget{ref-R-bindrcpp}{}
Müller, K. (2018). \emph{Bindrcpp: An 'rcpp' interface to active
bindings}. Retrieved from
\url{https://CRAN.R-project.org/package=bindrcpp}

\hypertarget{ref-R-base}{}
R Core Team. (2018). \emph{R: A language and environment for statistical
computing}. Vienna, Austria: R Foundation for Statistical Computing.
Retrieved from \url{https://www.R-project.org/}

\hypertarget{ref-R-psych}{}
Revelle, W. (2018). \emph{Psych: Procedures for psychological,
psychometric, and personality research}. Evanston, Illinois:
Northwestern University. Retrieved from
\url{https://CRAN.R-project.org/package=psych}

\hypertarget{ref-R-broom}{}
Robinson, D., \& Hayes, A. (2018). \emph{Broom: Convert statistical
analysis objects into tidy tibbles}. Retrieved from
\url{https://CRAN.R-project.org/package=broom}

\hypertarget{ref-R-apaTables}{}
Stanley, D. (2018). \emph{ApaTables: Create american psychological
association (apa) style tables}. Retrieved from
\url{https://CRAN.R-project.org/package=apaTables}

\hypertarget{ref-R-haven}{}
Wickham, H., \& Miller, E. (2018). \emph{Haven: Import and export
'spss', 'stata' and 'sas' files}. Retrieved from
\url{https://CRAN.R-project.org/package=haven}

\hypertarget{ref-R-dplyr}{}
Wickham, H., François, R., Henry, L., \& Müller, K. (2018). \emph{Dplyr:
A grammar of data manipulation}. Retrieved from
\url{https://CRAN.R-project.org/package=dplyr}

\hypertarget{ref-R-knitr}{}
Xie, Y. (2015). \emph{Dynamic documents with R and knitr} (2nd ed.).
Boca Raton, Florida: Chapman; Hall/CRC. Retrieved from
\url{https://yihui.name/knitr/}

\endgroup


\end{document}
